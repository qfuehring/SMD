\section{Aufgabe 1}

\subsection{1 a)}
Implementierung im Python-Skript.

\begin{figure}[H]
\centering
\includegraphics[scale=.7]{SP_OvQ.png}
\caption{Der SalePrice in Abhängigkeit von der OverallQuality. In rot der Fit aus Aufgabenteil b).}
\end{figure}


\begin{figure}[H]
\centering
\includegraphics[scale=.7]{SP_GLA.png}
\caption{Der SalePrice in Abhängigkeit von der OverallQuality.}
\end{figure}


\begin{figure}[H]
\centering
\includegraphics[scale=.7]{SP_GC.png}
\caption{Der SalePrice in Abhängigkeit von den Garagenplätzen.}
\end{figure}

\subsection{1 c)}

\begin{figure}[H]
\centering
\includegraphics[scale=.7]{Diff.png}
\caption{Die Differenz vom gefitteten Preis und den reellen Einzelpreisen.}
\end{figure}

\subsection{1 d)}

Lösung in der angehängten Datei \textit{foo.csv}.
