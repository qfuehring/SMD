\section*{Aufgabe 1}
\subsection*{a)}
\textit{Die Aufgabe kann analytisch gelöst werden, wurde allerdings auch als Pythondatei angehängt.} \\~\\
Zu den gegebenen Datenpunkten~$k_\text{i}\in(8,9,13)$ soll eine Poisson-Verteilung gefunden werden. Diese ist folgendermaßen definiert:

\begin{equation*}
P(k, \lambda) = \frac{\lambda^k}{k!}\cdot\mathrm{e}^{-\lambda}
\end{equation*}

Zu dieser Funktion kann die negative log-Likelihoodfunktion definiert werden:

\begin{equation*}
-\sum_{i=1}^3 \mathrm{ln}(P(k_i,\lambda)) = -( \mathrm{ln}(\frac{\lambda^8}{8!}\cdot\mathrm{e}^{-\lambda})+
\mathrm{ln}(\frac{\lambda^9}{9!}\cdot\mathrm{e}^{-\lambda})+
\mathrm{ln}(\frac{\lambda^{13}}{13!}\cdot\mathrm{e}^{-\lambda}))
\end{equation*}

mit Logarithmusgesetzen folgt:
\begin{equation*}
-\sum_{i=1}^3 \mathrm{ln}(P(k_i,\lambda)) = +3\cdot\lambda -\mathrm{ln}\left(\frac{\lambda^8}{8!}\right)-\mathrm{ln}\left(\frac{\lambda^9}{9!}\right) -\mathrm{ln}\left(\frac{\lambda^{13}}{13!}\right)
\end{equation*}

\begin{equation*}
-\sum_{i=1}^3 \mathrm{ln}(P(k_i,\lambda)) = +3\cdot\lambda -\mathrm{ln}(\lambda^{30})+\mathrm{ln}(8!9!13!)
\end{equation*}

\begin{equation*}
-\sum_{i=1}^3 \mathrm{ln}(P(k_i,\lambda)) = +3\cdot\lambda -30\cdot\mathrm{ln}(\lambda)+\mathrm{ln}(8!9!13!)
\end{equation*}

\subsection*{b)}
Diese kann nun minimiert werden:


\begin{equation*}
\frac{\mathrm{d}}{\mathrm{d}\lambda}\sum_{i=1}^3 \mathrm{ln}(P(k_i,\lambda)) = -3+\frac{8}{\lambda}+\frac{9}{\lambda}+\frac{13}{\lambda} \stackrel{!}{=} 0
\end{equation*}
\begin{equation*}
\lambda = 10
\end{equation*}

\subsection*{c)}
\begin{table}[H]
\centering
\begin{tabular}{ll}
$\lambda_1$ = 8,2836 & erster $\sigma$-Bereich\\
$\lambda_2$ = 11,9385 &\\
$\lambda_3$ = 6,7788 & zweiter $\sigma$-Bereich\\
$\lambda_4$ = 14,1089& \\
$\lambda_5$ = 5,4737 & dritter $\sigma$-Bereich\\
$\lambda_6$ = 16,5197&
\end{tabular}
\caption{Die zu berechnenden Werte}
\end{table}

\noindent
Die Werte ergeben sich aus der Formel~(\ref{eq:1}) für die Berechnung des nten $\sigma$-Fehlerintervalls bei der Maximumlikelihood-Methode.

\begin{equation}
F = F_\text{min}+\frac{n^2}{2}
\label{eq:1}
\end{equation}

\subsection*{d)}
