\newpage\section*{Aufgabe 3}
\subsection*{a}
\begin{equation}
I(p,n) = -\frac{p}{p+n}\log_2\biggl(\frac{p}{p+n}\biggr)-\frac{n}{p+n}\log_2\biggl(\frac{n}{p+n}\biggr)
\end{equation}
\begin{align*}
& P := \text{Fußball}&\Rightarrow & p = 9\\
& N := \overline{\text{Fußball}}&\Rightarrow & n = 5
\end{align*}
\begin{equation*}
I(p=9,n=5) \approx 0.94
\end{equation*}

\subsection*{b}
\begin{align}
\text{gain}(a) &= I(p,n) - E(a)\\
E(a) &= \sum_i \frac{p_i+n_i}{p+n}I(p_i,n_i)
\end{align}
\begin{align*}
& i = 1 &\Rightarrow & a := \text{Wind}\\
&&\Rightarrow & p_1 := 3\\
&&\Rightarrow & n_1 := 3\\
&&\Rightarrow & I(p_1=3,n_1=3) = 1\\\\
& i = 0 &\Rightarrow &  a := \overline{\text{Wind}}\\
&&\Rightarrow & p_1 := 6\\
&&\Rightarrow & n_1 := 2\\
&&\Rightarrow & I(p_1=6,n_1=2) = \approx 0.89\\
\end{align*}
\begin{align*}
E(\text{Wind}) &= \frac{8}{14}I(p_0=6,n_0=2) + \frac{6}{14}I(p_1=3,n_1=3)&\approx 0.89\\
\text{gain}(\text{Wind}) &= I(p,n) - E(\text{Wind}) &\approx 5\%\\
\end{align*}
\subsection*{c}
\includegraphics[width=0.9\linewidth]{../build/Temperatur}\newline
\includegraphics[width=0.9\linewidth]{../build/Wettervorhersage}\newline
\includegraphics[width=0.9\linewidth]{../build/Luftfeuchtigkeit}\newline
\includegraphics[width=0.9\linewidth]{../build/Wind}\newline
\subsection*{d}
Die höchste Trennkraft bei einem einzigen rechtwinkligen Schnitt auf ein einzelnes Attribut hat ein Schnitt auf eine Wettervorhersage der Klasse 1 mit einem Informationsgewinn von $\text{gain}(\text{Wettervorhersage}==1)=22.6\%$.

