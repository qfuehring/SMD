\newpage\section*{Aufgabe 1}
\subsection*{1a)}
Liegen die Attribute, die durch einen k-NN-Algorithmus klassifiziert werden sollen mehrere Größenordnungen auseinander, muss auf die Wahl des richtigen Abstandmaßes geachtet werden. Als Beispiel kann eine hochenergetische Messung betrachtet werden:

\begin{equation*}
\text{Messung:}(\text{Population 1:}\hspace{0.5cm}10~\text{GeV},8~\text{GeV}, 4~\text{GeV},\text{Population 2:}\hspace{0.5cm} 5~\text{eV}, 10~\text{eV})
\end{equation*}
\noindent
Der Algorithmus würde anhand des euklidischen Maßes die 4~GeV Messung in die Messwerte der falschen Größenordnung zuordnen. Das euklidische Maß wäre hier weniger geeignet.

\subsection*{1b)}
Da der gesamte Lernprozess während des Vorgangs geschieht und nicht wie bei einem Tree offline, benötigt der k-NN-Algorithmus verhältnismäßig lange für die Trainingsphase.

\subsection*{1c)}
Implementierung des k-NN-Algorithmus im Pythonskript.
\textbf{ACHTUNG: Zum Ausführen des Skripts bitte das Rootfile von der Homepage in den gleichen Ordner legen!! War zu groß zum Upload.}
\subsection*{1d)}
