\documentclass[
  bibliography=totoc,     % Literatur im Inhaltsverzeichnis
  captions=tableheading,  % Tabellenüberschriften
  titlepage=firstiscover, % Titelseite ist Deckblatt
]{scrartcl}

\usepackage{longtable}
\usepackage{multirow}
% LaTeX2e korrigieren.
\usepackage{fixltx2e}
% Warnung, falls nochmal kompiliert werden muss
\usepackage[aux]{rerunfilecheck}

% deutsche Spracheinstellungen
\usepackage{polyglossia}
\setmainlanguage{german}

% unverzichtbare Mathe-Befehle
\usepackage{amsmath}
% viele Mathe-Symbole
\usepackage{amssymb}
% Erweiterungen für amsmath
\usepackage{mathtools}

% Fonteinstellungen
\usepackage{fontspec}
\defaultfontfeatures{Ligatures=TeX}

\usepackage[
  math-style=ISO,    % \
  bold-style=ISO,    % |
  sans-style=italic, % | ISO-Standard folgen
  nabla=upright,     % |
  partial=upright,   % /
]{unicode-math}

\setmathfont{Latin Modern Math}
\setmathfont[range={\mathscr, \mathbfscr}]{XITS Math}

% make bar horizontal, use \hslash for slashed h
\let\hbar\relax
\DeclareMathSymbol{\hbar}{\mathord}{AMSb}{"7E}
\DeclareMathSymbol{ℏ}{\mathord}{AMSb}{"7E}

% richtige Anführungszeichen
\usepackage[autostyle]{csquotes}

% Zahlen und Einheiten
\usepackage[
  locale=DE,                   % deutsche Einstellungen
  separate-uncertainty=true,   % Immer Fehler mit \pm
  per-mode=symbol-or-fraction, % m/s im Text, sonst Brüche
]{siunitx}

% chemische Formeln
\usepackage[version=3]{mhchem}

% schöne Brüche im Text
\usepackage{xfrac}

% Floats innerhalb einer Section halten
\usepackage[section, below]{placeins}
% Captions schöner machen.
\usepackage[
  labelfont=bf,        % Tabelle x: Abbildung y: ist jetzt fett
  font=small,          % Schrift etwas kleiner als Dokument
  width=0.9\textwidth, % maximale Breite einer Caption schmaler
]{caption}
% subfigure, subtable, subref
\usepackage{subcaption}

% Grafiken können eingebunden werden
\usepackage{graphicx}
% größere Variation von Dateinamen möglich
\usepackage{grffile}

% Standardplatzierung für Floats einstellen
\usepackage{float}
\floatplacement{figure}{htbp}
\floatplacement{table}{htbp}

% schöne Tabellen
\usepackage{booktabs}

% Seite drehen für breite Tabellen
\usepackage{pdflscape}

% Literaturverzeichnis
\usepackage{biblatex}
% Quellendatenbank
\addbibresource{lit.bib}
%\addbibresource{programme.bib}

% Hyperlinks im Dokument
\usepackage[
  unicode,
  pdfusetitle,    % Titel, Autoren und Datum als PDF-Attribute
  pdfcreator={},  % PDF-Attribute säubern
  pdfproducer={}, % "
]{hyperref}
% erweiterte Bookmarks im PDF
\usepackage{bookmark}

% Trennung von Wörtern mit Strichen
\usepackage[shortcuts]{extdash}

\author{
  Robin Eichhorn%
  \texorpdfstring{
    \\
    \href{mailto:robin.eichhorn@tu-dortmund.de}{robin.eichhorn@tu-dortmund.de}
  }{}%
  \texorpdfstring{\and}{, }
    Maximilian Ernst
    \texorpdfstring{
      \\
      \href{mailto:maximilian.ernst@tu-dortmund.de}{maximilian.ernst@tu-dortmund.de}
    }{}
  \texorpdfstring{\and}{, }
  Marc Quentin Führing
  \texorpdfstring{
    \\
    \href{mailto:quentin.fuehring@tu-dortmund.de}{quentin.fuehring@tu-dortmund.de}
  }{}
}
\publishers{TU Dortmund – Fakultät Physik}
