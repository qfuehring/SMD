\documentclass[
  bibliography=totoc,     % Literatur im Inhaltsverzeichnis
  captions=tableheading,  % Tabellenüberschriften
  titlepage=firstiscover, % Titelseite ist Deckblatt
]{scrartcl}

\usepackage{longtable}
\usepackage{multirow}
% LaTeX2e korrigieren.
\usepackage{fixltx2e}
% Warnung, falls nochmal kompiliert werden muss
\usepackage[aux]{rerunfilecheck}

% deutsche Spracheinstellungen
\usepackage{polyglossia}
\setmainlanguage{german}

% unverzichtbare Mathe-Befehle
\usepackage{amsmath}
% viele Mathe-Symbole
\usepackage{amssymb}
% Erweiterungen für amsmath
\usepackage{mathtools}

% Fonteinstellungen
\usepackage{fontspec}
\defaultfontfeatures{Ligatures=TeX}

\usepackage[
  math-style=ISO,    % \
  bold-style=ISO,    % |
  sans-style=italic, % | ISO-Standard folgen
  nabla=upright,     % |
  partial=upright,   % /
]{unicode-math}

\setmathfont{Latin Modern Math}
\setmathfont[range={\mathscr, \mathbfscr}]{XITS Math}

% make bar horizontal, use \hslash for slashed h
\let\hbar\relax
\DeclareMathSymbol{\hbar}{\mathord}{AMSb}{"7E}
\DeclareMathSymbol{ℏ}{\mathord}{AMSb}{"7E}

% richtige Anführungszeichen
\usepackage[autostyle]{csquotes}

% Zahlen und Einheiten
\usepackage[
  locale=DE,                   % deutsche Einstellungen
  separate-uncertainty=true,   % Immer Fehler mit \pm
  per-mode=symbol-or-fraction, % m/s im Text, sonst Brüche
]{siunitx}

% chemische Formeln
\usepackage[version=3]{mhchem}

% schöne Brüche im Text
\usepackage{xfrac}

% Floats innerhalb einer Section halten
\usepackage[section, below]{placeins}
% Captions schöner machen.
\usepackage[
  labelfont=bf,        % Tabelle x: Abbildung y: ist jetzt fett
  font=small,          % Schrift etwas kleiner als Dokument
  width=0.9\textwidth, % maximale Breite einer Caption schmaler
]{caption}
% subfigure, subtable, subref
\usepackage{subcaption}

% Grafiken können eingebunden werden
\usepackage{graphicx}
% größere Variation von Dateinamen möglich
\usepackage{grffile}

% Standardplatzierung für Floats einstellen
\usepackage{float}
\floatplacement{figure}{htbp}
\floatplacement{table}{htbp}

% schöne Tabellen
\usepackage{booktabs}

% Seite drehen für breite Tabellen
\usepackage{pdflscape}

% Literaturverzeichnis
\usepackage{biblatex}
% Quellendatenbank
\addbibresource{lit.bib}
%\addbibresource{programme.bib}

% Hyperlinks im Dokument
\usepackage[
  unicode,
  pdfusetitle,    % Titel, Autoren und Datum als PDF-Attribute
  pdfcreator={},  % PDF-Attribute säubern
  pdfproducer={}, % "
]{hyperref}
% erweiterte Bookmarks im PDF
\usepackage{bookmark}

% Trennung von Wörtern mit Strichen
\usepackage[shortcuts]{extdash}

\author{
  Robin Eichhorn%
  \texorpdfstring{
    \\
    \href{mailto:robin.eichhorn@tu-dortmund.de}{robin.eichhorn@tu-dortmund.de}
  }{}%
  \texorpdfstring{\and}{, }
    Maximilian Ernst
    \texorpdfstring{
      \\
      \href{mailto:maximilian.ernst@tu-dortmund.de}{maximilian.ernst@tu-dortmund.de}
    }{}
  \texorpdfstring{\and}{, }
  Marc Quentin Führing
  \texorpdfstring{
    \\
    \href{mailto:quentin.fuehring@tu-dortmund.de}{quentin.fuehring@tu-dortmund.de}
  }{}
}
\publishers{TU Dortmund – Fakultät Physik}


%%% Hier definiert man Titel, Autor und Datum %%%%%%%%%%%%%%%%%%%%%%%%%%%%%%%%%

\subject{SMD}
\title{Übungsblatt 01}
\date{
  Bearbeitung: 02.-08. November 2016
  \newline
  Abgabe: 08. November 2016
}

%%%%%%%%%%%%%%%%%%%%%%%%%%%%%%%%%%%%%%%%%%%%%%%%%%%%%%%%%%%%%%%%%%%%%%%%%%%%%%%

\begin{document}

\maketitle
\section*{Aufgabe 1}
\subsection*{a)}
\textit{Die Aufgabe kann analytisch gelöst werden, wurde allerdings auch als Pythondatei angehängt.} \\~\\
Zu den gegebenen Datenpunkten~$k_\text{i}\in(8,9,13)$ soll eine Poisson-Verteilung gefunden werden. Diese ist folgendermaßen definiert:

\begin{equation*}
P(k, \lambda) = \frac{\lambda^k}{k!}\cdot\mathrm{e}^{-\lambda}
\end{equation*}

Zu dieser Funktion kann die negative log-Likelihoodfunktion definiert werden:

\begin{equation*}
-\sum_{i=1}^3 \mathrm{\ln}(P(k_i,\lambda)) = -( \mathrm{ln}(\frac{\lambda^8}{8!}\cdot\mathrm{e}^{-\lambda})+
\mathrm{ln}(\frac{\lambda^9}{9!}\cdot\mathrm{e}^{-\lambda})+
\mathrm{ln}(\frac{\lambda^{13}}{13!}\cdot\mathrm{e}^{-\lambda}))
\end{equation*}

mit Logarithmusgesetzen folgt:
\begin{equation*}
-\sum_{i=1}^3 \mathrm{ln}(P(k_i,\lambda)) = +3\cdot\lambda -\mathrm{ln}\left(\frac{\lambda^8}{8!}\right)-\mathrm{ln}\left(\frac{\lambda^9}{9!}\right) -\mathrm{ln}\left(\frac{\lambda^{13}}{13!}\right)
\end{equation*}

\begin{equation*}
-\sum_{i=1}^3 \mathrm{ln}(P(k_i,\lambda)) = +3\cdot\lambda -\mathrm{ln}(\lambda^{30})+\mathrm{ln}(8!9!13!)
\end{equation*}

\begin{equation*}
-\sum_{i=1}^3 \mathrm{ln}(P(k_i,\lambda)) = +3\cdot\lambda -30\cdot\mathrm{ln}(\lambda)+\mathrm{ln}(8!9!13!)
\end{equation*}

\begin{figure}[H]
  \centering
  \includegraphics[scale=.4]{negLog.pdf}
  \caption{Die berechnete negLogLikelihoodfunktion.}
\end{figure}

\subsection*{b)}
Diese kann nun minimiert werden:


\begin{equation*}
\frac{\mathrm{d}}{\mathrm{d}\lambda}\sum_{i=1}^3 \mathrm{ln}(P(k_i,\lambda)) = -3+\frac{8}{\lambda}+\frac{9}{\lambda}+\frac{13}{\lambda} \stackrel{!}{=} 0
\end{equation*}
\begin{equation*}
\lambda = 10
\end{equation*}

\subsection*{c)}
\begin{table}[H]
\centering
\begin{tabular}{ll}
$\lambda_1$ = 8,2836 & erster $\sigma$-Bereich\\
$\lambda_2$ = 11,9385 &\\
$\lambda_3$ = 6,7788 & zweiter $\sigma$-Bereich\\
$\lambda_4$ = 14,1089& \\
$\lambda_5$ = 5,4737 & dritter $\sigma$-Bereich\\
$\lambda_6$ = 16,5197&
\end{tabular}
\caption{Die zu berechnenden Werte}
\end{table}

\noindent
Die Werte ergeben sich aus der Formel~(\ref{eq:1}) für die Berechnung des nten $\sigma$-Fehlerintervalls bei der Maximumlikelihood-Methode.

\begin{equation}
F = F_\text{min}+\frac{n^2}{2}
\label{eq:1}
\end{equation}

\subsection*{d)}

Die Taylorentwicklung zweiter Ordnung der negativen log-Likelihoodfunktion ergibt sich zu:

%\begin{equation}
%
%\end{equation}
\noindent
Mit der sich die folgenden Werte analog zu c berechenen.
\begin{table}[H]
\centering
\begin{tabular}{ll}
$\lambda_1$ = 8,2836 & erster $\sigma$-Bereich\\
$\lambda_2$ = 11,9385 &\\
$\lambda_3$ = 6,7788 & zweiter $\sigma$-Bereich\\
$\lambda_4$ = 14,1089& \\
$\lambda_5$ = 5,4737 & dritter $\sigma$-Bereich\\
$\lambda_6$ = 16,5197&
\end{tabular}
\caption{Mit der Taylorentwicklung berechnete Werte analog zu c)}
\end{table}
\noindent
Die Taylorentwicklung und die ursprüngliche negative log-Likelihoodfunktion werden in Abbildung~\ref{d} dargestellt.

\begin{figure}[H]
\centering
\includegraphics[scale=.4]{negLog.pdf}
\caption{Vergleich zwischen der Taylorentwicklung und der negativen log-Likelihood}
\label{d}
\end{figure}

\section*{Aufg 2}
\includepdf[pages={1-3}]{aufg2.pdf}
\section{Aufgabe 3}
\hspace{-4cm}
\includegraphics[width=1.5\linewidth]{./Blatt4/Blatt4_3_1 001}}
\newpage
\hspace{-4cm}
\includegraphics[width=1.5\linewidth]{./Blatt4/Blatt4_3_2 001}}
\newpage
\hspace{-4cm}
\includegraphics[width=1.5\linewidth]{./Blatt4/Blatt4_3_3 001}}


\newpage\section{Aufgabe 4}
\subsection{a)}
Ohne Korrelation:
\begin{align}
\sigma_y &= \sqrt{\biggl( \frac{\partial y}{\partial a_0}\cdot \sigma_{a_0}\biggr)^2 
+\biggl( \frac{\partial y}{\partial a_1}\cdot \sigma_{a_1}\biggr)^2}\\
&= \sqrt{( \sigma_{a_0})^2 
+ ( x^2\cdot \sigma_{a_1})^2}\\
&= \sqrt{\SI{0.04}{} + x^2 \cdot \SI{0.04}{}}\\
&= \SI{0.2}{}\sqrt{1+x^2}
\end{align}
Mit Korrelation:
\begin{align}
\sigma_y &= \sqrt{\biggl( \frac{\partial y}{\partial a_0}\cdot \sigma_{a_0}\biggr)^2 
+\biggl( \frac{\partial y}{\partial a_1}\cdot \sigma_{a_1}\biggr)^2
+2\cdot\biggl(\frac{\partial y}{\partial a_0}\biggr)\biggl(\frac{\partial y}{\partial a_1}\biggr)\cdot \text{cov}(a_0,a_1)}\\
&= \sqrt{( \sigma_{a_0})^2 
+ ( x^2\cdot \sigma_{a_1})^2
+ 2x \rho\sigma_{a_0}\sigma_{a_1}}\\
&= \sqrt{\SI{0.04}{} + x^2 \cdot \SI{0.04}{} + 2x\cdot(\SI{-0.032}{})}\\
&= \SI{0.2}{}\sqrt{1+x(x-\SI{1.6}{})}
\end{align}
\subsection{b)}
\subsection{c)}
Analytisch:
\begin{align}
\mu_y &= \mu_{a_0} + x\cdot \mu_{a_1}\\
& = \SI{1}+ x\\
\sigma_y &= \SI{0.2}{}\sqrt{1+x(x-\SI{1.6}{})}\\ \\
y(\SI{-3}{})&=\SI{-2.0(8)}{}\\
y(\SI{0}{})&=\SI{1.0(2)}{}\\
y(\SI{3}{})&=\SI{4.0(5)}{}
\end{align}



\end{document}
